%%%%%%%%%%%%%%%%%%%%%%%%%%%%%%%%%%%%%%%%%
% "ModernCV" CV and Cover Letter
% LaTeX Template
% Version 1.11 (19/6/14)
%
% This template has been downloaded from:
% http://www.LaTeXTemplates.com
%
% Original author:
% Xavier Danaux (xdanaux@gmail.com)
%
% License:
% CC BY-NC-SA 3.0 (http://creativecommons.org/licenses/by-nc-sa/3.0/)
%
% Important note:
% This template requires the moderncv.cls and .sty files to be in the same 
% directory as this .tex file. These files provide the resume style and themes 
% used for structuring the document.
%
%%%%%%%%%%%%%%%%%%%%%%%%%%%%%%%%%%%%%%%%%

%----------------------------------------------------------------------------------------
%	PACKAGES AND OTHER DOCUMENT CONFIGURATIONS
%----------------------------------------------------------------------------------------

\documentclass[11pt,a4paper,sans]{moderncv} % Font sizes: 10, 11, or 12; paper sizes: a4paper, letterpaper, a5paper, legalpaper, executivepaper or landscape; font families: sans or roman

\moderncvstyle{casual} % CV theme - options include: 'casual' (default), 'classic', 'oldstyle' and 'banking'
\moderncvcolor{blue} % CV color - options include: 'blue' (default), 'orange', 'green', 'red', 'purple', 'grey' and 'black'

\usepackage[T1,T2A]{fontenc}
\usepackage[utf8]{inputenc}

\usepackage{lipsum} % Used for inserting dummy 'Lorem ipsum' text into the template

\usepackage[scale=0.75]{geometry} % Reduce document margins
%\setlength{\hintscolumnwidth}{3cm} % Uncomment to change the width of the dates column
%\setlength{\makecvtitlenamewidth}{10cm} % For the 'classic' style, uncomment to adjust the width of the space allocated to your name

%----------------------------------------------------------------------------------------
%	NAME AND CONTACT INFORMATION SECTION
%----------------------------------------------------------------------------------------

\firstname{Арсений} % Your first name
\familyname{Трушин} % Your last name

% All information in this block is optional, comment out any lines you don't need
\title{Резюме}
\address{Малая Сухаревская пл., 3, 128}{Москва, Россия 129090}
\mobile{+7(916) 588 5156}
\phone{+7(495) 694 1520}
%\fax{(000) 111 1113}
\email{tarstars@gmail.com}
\homepage{tarsenyss.narod.ru}{tarsenyss.narod.ru} % The first argument 
%is the url for the clickable link, the second argument is the url displayed in 
%the template - this allows special characters to be displayed such as the 
%tilde in this example
%\extrainfo{additional information}
\photo[70pt][0.4pt]{pictures/trushin} % The first bracket is the picture 
%height, the second is the thickness of the frame around the picture (0pt for 
%no frame)
%\quote{"A witty and playful quotation" - John Smith}

%----------------------------------------------------------------------------------------

\begin{document}

\makecvtitle % Print the CV title

%----------------------------------------------------------------------------------------
%	EDUCATION SECTION
%----------------------------------------------------------------------------------------

\section{Образование}
%\textit{GPA -- 4.5}
%Arguments not required %can 
%be left empty
\cventry{1996--2001}{Студент}{Физический факультет МГУ 
им М.В. Ломоносова}{Москва}{}{Специальность - физика}
\cventry{2001-2004}{Аспирант}{Физический факультет МГУ 
им. М.В. Ломоносова}{Москва}{}{Специальность - лазерная физика}

\section{Диплом к.ф.-м.н.}

\cvitem{Название}{\emph{Фотолюминесценция иттербия в полопроводниковых 
структурах и наноструктурах ZnSe-ZnCdSe}}
\cvitem{Научный руководитель}{Коннов В.М.}
\cvitem{Описание}{Были оптимизированы условия проведения ионной 
имплантации с последующим отжигом для получения максимально полного 
восстановления кристаллического совершенства материала и наибольшей 
интенсивности излучения в инфракрасном диапазоне ионов иттербия.}
\cvitem{Дата}{Март 2004}

%----------------------------------------------------------------------------------------
%	WORK EXPERIENCE SECTION
%----------------------------------------------------------------------------------------

\section{Опыт}

\subsection{Работы}

\cventry{2004--н.в.}{научный сотрудник}{\textsc{ФИАН им. 
П.Н. Лебедева}}{Москва}{}{Разработка программного обеспечения для 
автоматизации эксперимента нахождения C-V характеристик, 
проведение экспериментального исследования катодо- и фото- люминесценции 
полупроводниковых и алмазных структурю}
%------------------------------------------------

\cventry{2009--2014}{ассистент}{\textsc{Физический 
факультет МГУ им. М.В. Ломоносова}}{Москва}{}{Моделирование акустических и 
оптических процессов в кристаллах.}

%------------------------------------------------
\newpage

\subsection{Написанные программы}

\cventry{1996--2001}{Студент}{Физический факультет}{}{}{Несколько типичных для 
Физического Факультета МГУ проектов, таких как ``Эллипсоид инерции'', 
``Задача трёх тел'', ``Дифракция рентгеновского излучения на кристалле, 
деформированном ионной имплантацией'', ``Совмещение снимков сетчатки глаза''. 
В каждом из этих проектов было несколько сот строк кода. Это были программы 
под Windows с графическим интерфейсом пользователя. Keywords: win32api, 
mfc, numeric modeling, c++. }

\cventry{2001-2003}{Анализ рентгеновских спектров}{}{}{}{Разработка 
программного обеспечения для определения состава образца по спектру его 
рентгеновской люминесценции. Задачи: разработка технического задания на 
программу, математическое моделирование, разработка GUI, реализация. 
Keywords: win32api, mfc, c++. 
Примерное количество строк \(\approx\) 20000}


\cventry{2003-2004}{Анализ стихотворение}{}{}{}{Написание технического задания, 
разработка алгоритмов по приблизительным формулировкам заказчика, разработка 
GUI, Реализация. Keywords: win32api, mfc, c++, com, VBA. Примерное количество 
строк \(\approx\) 10000}

\cventry{2004-н.в.}{Пользователь линукса}{}{}{}{Я знаком с такими операционными 
системами как Fedora и Ubuntu. Последние семь лет использую Gentoo.}

\cventry{2005-2006}{Артезио}{}{}{}{Разработчик программного обеспечения в 
проекте фирмы Телкордия в фирме Артезио. Keywords: ssh, unix, c++, 
vim}

\cventry{2006-2009}{ФИАН}{}{}{}{Разработка нескольких инструментов для научных 
исследований, например, усреднение данных оптической профилометрии 
(C++, golang, \(\approx1000\) строк кода ); система навигации по звёздам 
(python, IDE, c++, asm)} 

\cventry{2009-2014}{Физический факультет МГУ им. М.В. 
Ломоносова}{}{}{}{Разработка программного обеспечения для 
вычислениях характеристик объёмных, однородных и неоднородных волн в 
анизотропных средах. Моделирование процессов возбуждения и 
распространения акустических пучков в кристаллах. Решаемые задачи: постановка 
задачи, разработка математической модели, разработка архитектуры программы, 
реализация. Keywords: Qt, STL, C++, golang, povray, 
opengl, git.} 

\cventry{2010}{Каталог статей}{}{}{}{Сервер на Python для отображения 
локального архива статей}
%----------------------------------------------------------------------------------------
%	AWARDS SECTION
%----------------------------------------------------------------------------------------

\section{Награды}

\cvitem{2004}{Премия Фени Берц}
\cvitem{2010}{``Лучший доклад конференции'' - Школа по когерентной оптике и 
оптической спектроскопии}

%----------------------------------------------------------------------------------------
%	COMPUTER SKILLS SECTION
%----------------------------------------------------------------------------------------

\section{Знание компьютерных технологий}

%\cvitem{Basic}{\textsc{java}, Adobe Illustrator}
\cvitem{Intermediate}{golang, python, git, win32api, mfc, bash, html}
\cvitem{Advanced}{C++}

%----------------------------------------------------------------------------------------
%	COMMUNICATION SKILLS SECTION
%----------------------------------------------------------------------------------------

\section{Навыки общения}

\cvitem{2000-2014}{Свыше 20 устных докладов на различных конференциях 
включая 4 международных конференции с докладами на Английском языке}

%-------------------------------------------------------------------------------
%---------
%	LANGUAGES SECTION
%----------------------------------------------------------------------------------------

\section{Languages}

\cvitemwithcomment{Russian}{Родной}{}
\cvitemwithcomment{English}{Intermediate}{Беглое чтение, беглый разговор}

%----------------------------------------------------------------------------------------
%	INTERESTS SECTION
%----------------------------------------------------------------------------------------

\section{Разное}
\cvitem{2010-2014}{Вёл курс ``C++ в задачах акустооптики и оптоэлектроники''}
\cvitem{Topcoder login}{142857}
\cvitem{Знаком с книгами}{ ``Программирование: теоремы и задачи'', А. Шень; 
``Алгоритмы: построение и анализ'', Кормен, Лейзерсон}

\section{Интересы}

\renewcommand{\listitemsymbol}{-~} % Changes the symbol used for lists

Люблю кататься на своём велосипеде с планетарной коробкой переключения передач. 
Ещё есть фотоаппарат Canon 1000d. Иногда даже им пользуюсь ;)

%----------------------------------------------------------------------------------------
%	COVER LETTER
%----------------------------------------------------------------------------------------

% To remove the cover letter, comment out this entire block

%\clearpage

%\recipient{HR Department}{Corporation\\123 Pleasant Lane\\12345 City, State} % 
%Letter recipient
%\date{\today} % Letter date
%\opening{Dear Sir or Madam,} % Opening greeting
%\closing{Sincerely yours,} % Closing phrase
%\enclosure[Attached]{curriculum vit\ae{}} % List of enclosed documents

%\makelettertitle % Print letter title

%\lipsum[1-3] % Dummy text

%\makeletterclosing % Print letter signature

%----------------------------------------------------------------------------------------

\end{document}